\newpage
\newgeometry{includemp=true,
             inner=2cm,outer=2cm,
             top=2cm,bottom=1.5cm,
             headsep=0.5cm,headheight=0.5cm,
             footskip=0.7cm,
             marginparsep=0cm,marginparwidth=0cm}
\thispagestyle{empty}
\bookmark[page=4, level=1]{版权页}

\centerline{\textbf{内容简介}}
\vspace{0.8em}
{
\zihao{-5}

本书比较全面、系统地介绍了矩阵的基本理论、方法及其应用,全书分上、下两篇,上篇为基础篇,下篇为应用篇,共8章,分别介绍了矩阵的几何理论(包括线性空间与线性算子,内积空间与等积变换),矩阵与若尔当标准形,矩阵的分解,赋范线性空间与矩阵范数,矩阵微积分及其应用,广义逆矩阵及其应用,儿类特殊矩阵与特殊积(如非负矩阵与正矩阵、循环矩阵与素矩阵、随机矩阵和双随机矩阵、单调矩阵、M矩阵与H矩阵、T矩阵与汉克尔矩阵以及克罗内克积、阿达马积与反积等),前7章每章均配有一定数量的习题.附录中还给出了15套模拟自测试题,所有习题和自测题(约1300题)的详细解答,即将由清华大学出版社另行出版.

本书可作为理工科大学各专业研究生的学位课程教材,也可作为理工科和师范类院校高年级本科生的选修课教材,并可供有关专业的教师和工程技术人员参考.

\vfill

\noindent\hc{版权所有,侵权必究.}

\vspace{2em}

\begin{minipage}{0.9\linewidth}
    \hc{图书在版编目(CIP)数据}

    \vspace{1em}

    矩阵分析:可视化方法 / J. Wang编著. —北京:还不知道哪个出版社, 2222

    ISBN 996-7-618-12345-7

    \Romannum{1}.\ding{172}矩… \quad \Romannum{2}.\ding{172}Wang… \quad \Romannum{3}.\ding{172}矩阵分析 \quad \Romannum{4}. \ding{172}O151.7$x$ \quad 

    中国版本图书馆CIP数据核字(2222)第777777号
\end{minipage}

\vfill

\noindent\hc{责任编辑:} 阿\quad 磊\quad 陈尼龙 \par
\noindent\hc{封面设计:} J. Wang \par
\noindent\hc{责任校对:} 李涤纶 \par
\noindent\hc{责任印制:} 谢纯棉

\vspace{2em}
\noindent\hc{出版发行:} 还不知道哪个出版社

\vspace{0.5em}
\noindent\sj[4.7]\begin{minipage}{0.8\linewidth}
    \hc{网\qquad 址:} \texttt{http://www.hbzd.press.cn}  \par
    \hc{地\qquad 址:} 北京还不知道那条路66号A座 \qquad \hc{邮\qquad 编:} 000000 \par
    \hc{投稿与读者服务:}    \par
    \hc{质量反馈:}         
\end{minipage}

\vspace{0.5em}
\noindent\hc{印\qquad 刷:} \par
\noindent\hc{装\qquad 订:} \par
\noindent\hc{经\qquad 销:} \par
\noindent\hc{开\qquad 本:} 8\,\si{in}$\times$10\,\si{in}\qquad \hc{印\qquad 张:} 27.7 \qquad \hc{字\qquad 数:}  777千字\par
\noindent\hc{版\qquad 次:} 2222年7月第1版 \qquad \hc{印\qquad 次:} 2222年7月第7次印刷\par
\noindent\hc{印\qquad 数:} $1\sim 7777$\par
\noindent\hc{定\qquad 价:} 77.77元 \par
\noindent\rule[1em]{\linewidth}{0.1pt}\par
\vspace{-1.5em}
\noindent 产品编号\hc{:} 098765-07
}
\newpage
\restoregeometry