\chapter{深度学习}

\section{测试}

我的第一个结果,是博士第一年的暑假开始做的,其实我是很幸运的,当时国内有位老师在我老板这里访问,这位老师非常热心并且耐心地教了我很多细节性的东西,让我能够快速上手做问题。很多细节性的东西,往往是需要对一个领域非常熟稔之后才能体悟到的,对于一个低年级的Phd来说,这时有人带着的话比自己看论文去学是省不少时间的,我真的很感激!

控制多图行距,临时改变行距.

\begin{figure}
  \begin{subfigure}[b]{.3\linewidth}
  \centering\large A
  \caption{A subfigure}\label{fig:1a}
  \end{subfigure}%
  \begin{subfigure}[b]{.3\linewidth}
  \centering\large B
  \caption{Another subfigure}\label{fig:1b}
  \end{subfigure} \quad
  \subcaptionbox[b]{A cat\label{cat}}
  [.3\linewidth]{\centering\large C}
  \caption{A figure}\label{fig:1}
\end{figure}


那个暑假我们做出了一个我们觉着很漂亮的结果,然后我就开始动手写了。写的过程,花费了很大的精力,毕竟是自己读Phd之后第一篇正儿八经的文章,中间就来来回回地思考怎么合理地安排证明结构,让证明能非常简洁漂亮,包括中间不断地修正证明里面的各种小漏洞。

因为我当时刚第二年,课业的压力还不小,博士是要求学一些本专业不是你研究方向的课,但因为我本科时并没有多少计算机基础,所以我上那些需要编程之类的课还是很耗精力的。加上当时我还做着助教,所以几乎都是挤时间在敲论文。

当时有一天夜里兴致来了,一直敲到三四点,把里面最关键的一个引理,11页的证明敲完了。我当时从系里往家走的时候,外面下着雪,还不小,可当时的我漫步雪中,觉着这雪飘得好温柔啊,一切都是那么的宁静和美好。回到家后,久久不能入睡,我忍不住发了下面这么一条朋友圈。那时的我,所感受到的是一种科研带来的发自内心深处的愉悦感和充实感。

这篇文章前前后后写了有半年,中间还因为准备qual exam(博士生资格考试)耽搁了一两个月,到了17年二月份的时候,差不多写好就准备投了。我老板也是非常负责上心,整篇文章,接近60页的证明,老板一步步地检查,给我写了密密麻麻的修改意见,从证明结构到单词语法,我其实挺感动的。

我觉着自己的博士生涯算是有了个不错的开端,希望自己能一直这样努力保持下去,最终收获一个还不错的结果。

那时的我不单单对学术,对自己的生活也是充满了憧憬与热爱。我办公室里养了很多花,我很细心地照料着她们,其中我最喜欢的是一盆兰花,“兰生幽谷,不以无人而不芳。”不是“孤芳自赏”之意,只是用于勉励自己在略显孤寂的博士生涯中能在各方面都努力提升自己。
